\documentclass{beamer}

\usepackage{graphicx,float,caption}
\usepackage{subcaption}
\usepackage{mdframed}
\usepackage{xcolor}
\usepackage{lipsum}
\usepackage{multimedia}
\usepackage{amsmath}
\usepackage[english]{babel}

\setbeamertemplate{footline}[frame number]

\usetheme{Warsaw}

\defbeamertemplate*{footline}{shadow theme}
{%
\leavevmode%
  \hbox{\begin{beamercolorbox}[wd=.5\paperwidth,ht=2.5ex,dp=1.125ex,leftskip=.3cm plus1fil,rightskip=.3cm]{author in head/foot}%
    \usebeamerfont{author in head/foot}\insertframenumber\,/\,\inserttotalframenumber\hfill\insertshortauthor
  \end{beamercolorbox}%
  \begin{beamercolorbox}[wd=.5\paperwidth,ht=2.5ex,dp=1.125ex,leftskip=.3cm,rightskip=.3cm plus1fil]{title in head/foot}%
    \usebeamerfont{title in head/foot}\insertshorttitle%
  \end{beamercolorbox}}%
  \vskip0pt%
}
\setbeamertemplate{caption}[numbered]

\title { Bayesian Reinforcement Learning Methods }
\subtitle { Using Bayesian MDPs and GPTD Methods }

\author[Vickie Ye and Alexandr Wang]
{ Vickie~Ye and Alexandr~Wang}

\date[Spring 2016]
{ May 12, 2016}

\begin{document}

\begin{frame}
\titlepage
\end{frame}

\begin{frame}{NMR measures the magnetic moment of nuclei}
\begin{minipage}{0.5\textwidth}
\begin{itemize}
\item A nucleus with magnetic moment $\mathbf{\mu} = \gamma \mathbf{I}$ in a magnetic field $\mathbf{B} = B_0 \hat{z} + B_1\sin\omega t \hat{x}$ precesses about the $z$ axis with Larmor frequency 
\begin{equation}
f = \gamma B_0.
\end{equation}
\item If we apply an RF pulse for time $\frac{\pi}{\gamma B_1}$, we can rotate the moment by $90^\circ$ into the $xy$ plane and observe the precessing magnetization. 
\end{itemize}
\end{minipage}%
\end{frame}

\end{document}
